
\documentclass{article}
%% ODER: format ==         = "\mathrel{==}"
%% ODER: format /=         = "\neq "
%
%
\makeatletter
\@ifundefined{lhs2tex.lhs2tex.sty.read}%
  {\@namedef{lhs2tex.lhs2tex.sty.read}{}%
   \newcommand\SkipToFmtEnd{}%
   \newcommand\EndFmtInput{}%
   \long\def\SkipToFmtEnd#1\EndFmtInput{}%
  }\SkipToFmtEnd

\newcommand\ReadOnlyOnce[1]{\@ifundefined{#1}{\@namedef{#1}{}}\SkipToFmtEnd}
\usepackage{amstext}
\usepackage{amssymb}
\usepackage{stmaryrd}
\DeclareFontFamily{OT1}{cmtex}{}
\DeclareFontShape{OT1}{cmtex}{m}{n}
  {<5><6><7><8>cmtex8
   <9>cmtex9
   <10><10.95><12><14.4><17.28><20.74><24.88>cmtex10}{}
\DeclareFontShape{OT1}{cmtex}{m}{it}
  {<-> ssub * cmtt/m/it}{}
\newcommand{\texfamily}{\fontfamily{cmtex}\selectfont}
\DeclareFontShape{OT1}{cmtt}{bx}{n}
  {<5><6><7><8>cmtt8
   <9>cmbtt9
   <10><10.95><12><14.4><17.28><20.74><24.88>cmbtt10}{}
\DeclareFontShape{OT1}{cmtex}{bx}{n}
  {<-> ssub * cmtt/bx/n}{}
\newcommand{\tex}[1]{\text{\texfamily#1}}	% NEU

\newcommand{\Sp}{\hskip.33334em\relax}


\newcommand{\Conid}[1]{\mathit{#1}}
\newcommand{\Varid}[1]{\mathit{#1}}
\newcommand{\anonymous}{\kern0.06em \vbox{\hrule\@width.5em}}
\newcommand{\plus}{\mathbin{+\!\!\!+}}
\newcommand{\bind}{\mathbin{>\!\!\!>\mkern-6.7mu=}}
\newcommand{\rbind}{\mathbin{=\mkern-6.7mu<\!\!\!<}}% suggested by Neil Mitchell
\newcommand{\sequ}{\mathbin{>\!\!\!>}}
\renewcommand{\leq}{\leqslant}
\renewcommand{\geq}{\geqslant}
\usepackage{polytable}

%mathindent has to be defined
\@ifundefined{mathindent}%
  {\newdimen\mathindent\mathindent\leftmargini}%
  {}%

\def\resethooks{%
  \global\let\SaveRestoreHook\empty
  \global\let\ColumnHook\empty}
\newcommand*{\savecolumns}[1][default]%
  {\g@addto@macro\SaveRestoreHook{\savecolumns[#1]}}
\newcommand*{\restorecolumns}[1][default]%
  {\g@addto@macro\SaveRestoreHook{\restorecolumns[#1]}}
\newcommand*{\aligncolumn}[2]%
  {\g@addto@macro\ColumnHook{\column{#1}{#2}}}

\resethooks

\newcommand{\onelinecommentchars}{\quad-{}- }
\newcommand{\commentbeginchars}{\enskip\{-}
\newcommand{\commentendchars}{-\}\enskip}

\newcommand{\visiblecomments}{%
  \let\onelinecomment=\onelinecommentchars
  \let\commentbegin=\commentbeginchars
  \let\commentend=\commentendchars}

\newcommand{\invisiblecomments}{%
  \let\onelinecomment=\empty
  \let\commentbegin=\empty
  \let\commentend=\empty}

\visiblecomments

\newlength{\blanklineskip}
\setlength{\blanklineskip}{0.66084ex}

\newcommand{\hsindent}[1]{\quad}% default is fixed indentation
\let\hspre\empty
\let\hspost\empty
\newcommand{\NB}{\textbf{NB}}
\newcommand{\Todo}[1]{$\langle$\textbf{To do:}~#1$\rangle$}

\EndFmtInput
\makeatother
%
%
%
%
%
%
% This package provides two environments suitable to take the place
% of hscode, called "plainhscode" and "arrayhscode". 
%
% The plain environment surrounds each code block by vertical space,
% and it uses \abovedisplayskip and \belowdisplayskip to get spacing
% similar to formulas. Note that if these dimensions are changed,
% the spacing around displayed math formulas changes as well.
% All code is indented using \leftskip.
%
% Changed 19.08.2004 to reflect changes in colorcode. Should work with
% CodeGroup.sty.
%
\ReadOnlyOnce{polycode.fmt}%
\makeatletter

\newcommand{\hsnewpar}[1]%
  {{\parskip=0pt\parindent=0pt\par\vskip #1\noindent}}

% can be used, for instance, to redefine the code size, by setting the
% command to \small or something alike
\newcommand{\hscodestyle}{}

% The command \sethscode can be used to switch the code formatting
% behaviour by mapping the hscode environment in the subst directive
% to a new LaTeX environment.

\newcommand{\sethscode}[1]%
  {\expandafter\let\expandafter\hscode\csname #1\endcsname
   \expandafter\let\expandafter\endhscode\csname end#1\endcsname}

% "compatibility" mode restores the non-polycode.fmt layout.

\newenvironment{compathscode}%
  {\par\noindent
   \advance\leftskip\mathindent
   \hscodestyle
   \let\\=\@normalcr
   \let\hspre\(\let\hspost\)%
   \pboxed}%
  {\endpboxed\)%
   \par\noindent
   \ignorespacesafterend}

\newcommand{\compaths}{\sethscode{compathscode}}

% "plain" mode is the proposed default.
% It should now work with \centering.
% This required some changes. The old version
% is still available for reference as oldplainhscode.

\newenvironment{plainhscode}%
  {\hsnewpar\abovedisplayskip
   \advance\leftskip\mathindent
   \hscodestyle
   \let\hspre\(\let\hspost\)%
   \pboxed}%
  {\endpboxed%
   \hsnewpar\belowdisplayskip
   \ignorespacesafterend}

\newenvironment{oldplainhscode}%
  {\hsnewpar\abovedisplayskip
   \advance\leftskip\mathindent
   \hscodestyle
   \let\\=\@normalcr
   \(\pboxed}%
  {\endpboxed\)%
   \hsnewpar\belowdisplayskip
   \ignorespacesafterend}

% Here, we make plainhscode the default environment.

\newcommand{\plainhs}{\sethscode{plainhscode}}
\newcommand{\oldplainhs}{\sethscode{oldplainhscode}}
\plainhs

% The arrayhscode is like plain, but makes use of polytable's
% parray environment which disallows page breaks in code blocks.

\newenvironment{arrayhscode}%
  {\hsnewpar\abovedisplayskip
   \advance\leftskip\mathindent
   \hscodestyle
   \let\\=\@normalcr
   \(\parray}%
  {\endparray\)%
   \hsnewpar\belowdisplayskip
   \ignorespacesafterend}

\newcommand{\arrayhs}{\sethscode{arrayhscode}}

% The mathhscode environment also makes use of polytable's parray 
% environment. It is supposed to be used only inside math mode 
% (I used it to typeset the type rules in my thesis).

\newenvironment{mathhscode}%
  {\parray}{\endparray}

\newcommand{\mathhs}{\sethscode{mathhscode}}

% texths is similar to mathhs, but works in text mode.

\newenvironment{texthscode}%
  {\(\parray}{\endparray\)}

\newcommand{\texths}{\sethscode{texthscode}}

% The framed environment places code in a framed box.

\def\codeframewidth{\arrayrulewidth}
\RequirePackage{calc}

\newenvironment{framedhscode}%
  {\parskip=\abovedisplayskip\par\noindent
   \hscodestyle
   \arrayrulewidth=\codeframewidth
   \tabular{@{}|p{\linewidth-2\arraycolsep-2\arrayrulewidth-2pt}|@{}}%
   \hline\framedhslinecorrect\\{-1.5ex}%
   \let\endoflinesave=\\
   \let\\=\@normalcr
   \(\pboxed}%
  {\endpboxed\)%
   \framedhslinecorrect\endoflinesave{.5ex}\hline
   \endtabular
   \parskip=\belowdisplayskip\par\noindent
   \ignorespacesafterend}

\newcommand{\framedhslinecorrect}[2]%
  {#1[#2]}

\newcommand{\framedhs}{\sethscode{framedhscode}}

% The inlinehscode environment is an experimental environment
% that can be used to typeset displayed code inline.

\newenvironment{inlinehscode}%
  {\(\def\column##1##2{}%
   \let\>\undefined\let\<\undefined\let\\\undefined
   \newcommand\>[1][]{}\newcommand\<[1][]{}\newcommand\\[1][]{}%
   \def\fromto##1##2##3{##3}%
   \def\nextline{}}{\) }%

\newcommand{\inlinehs}{\sethscode{inlinehscode}}

% The joincode environment is a separate environment that
% can be used to surround and thereby connect multiple code
% blocks.

\newenvironment{joincode}%
  {\let\orighscode=\hscode
   \let\origendhscode=\endhscode
   \def\endhscode{\def\hscode{\endgroup\def\@currenvir{hscode}\\}\begingroup}
   %\let\SaveRestoreHook=\empty
   %\let\ColumnHook=\empty
   %\let\resethooks=\empty
   \orighscode\def\hscode{\endgroup\def\@currenvir{hscode}}}%
  {\origendhscode
   \global\let\hscode=\orighscode
   \global\let\endhscode=\origendhscode}%

\makeatother
\EndFmtInput
%
\begin{document}

\section{Quick Sample!}

Might as well start with factorial up top as a demonstration of a function.

\begin{hscode}\SaveRestoreHook
\column{B}{@{}>{\hspre}l<{\hspost}@{}}%
\column{3}{@{}>{\hspre}l<{\hspost}@{}}%
\column{E}{@{}>{\hspre}l<{\hspost}@{}}%
\>[3]{}\Varid{fact}\mathbin{::}\Conid{Integer}\to \Conid{Integer}{}\<[E]%
\\
\>[3]{}\Varid{fact}\;\mathrm{1}\mathrel{=}\mathrm{1}{}\<[E]%
\\
\>[3]{}\Varid{fact}\;\Varid{n}\mathrel{=}\Varid{n}\mathbin{*}\Varid{fact}\;(\Varid{n}\mathbin{-}\mathrm{1}){}\<[E]%
\ColumnHook
\end{hscode}\resethooks
\section{On with the first chapter!}

So preliminaries...  Order of precedence is as you would expect, except
function application comes before anything else and doesn't require parens. 

\begin{hscode}\SaveRestoreHook
\column{B}{@{}>{\hspre}l<{\hspost}@{}}%
\column{3}{@{}>{\hspre}l<{\hspost}@{}}%
\column{E}{@{}>{\hspre}l<{\hspost}@{}}%
\>[3]{}\Varid{f}\;\mathrm{10}\mbox{\onelinecomment  means apply f to 10
}{}\<[E]%
\\
\>[3]{}\Varid{f}\;\mathrm{10}\mathbin{+}\mathrm{2}\mbox{\onelinecomment  is f of 10, then add 2
}{}\<[E]%
\\
\>[3]{}\Varid{f}\;(\mathrm{10}\mathbin{+}\mathrm{2})\mbox{\onelinecomment  is f of 12.
}{}\<[E]%
\\
\>[3]{}\Varid{f}\;\mathrm{10}\mathbin{+}\mathrm{2}\not\equiv \Varid{f}\;(\mathrm{10}\mathbin{+}\mathrm{2})\mbox{\onelinecomment  because function application is highest precedence
}{}\<[E]%
\ColumnHook
\end{hscode}\resethooks
Be careful with negative numbers!  
You may want to surround them with parentheses to play it safe. \ensuremath{\Varid{f}\;(\mathbin{-}\mathrm{3})}
\\

Boolean Algebra is simple: 

\begin{hscode}\SaveRestoreHook
\column{B}{@{}>{\hspre}l<{\hspost}@{}}%
\column{3}{@{}>{\hspre}l<{\hspost}@{}}%
\column{18}{@{}>{\hspre}l<{\hspost}@{}}%
\column{E}{@{}>{\hspre}l<{\hspost}@{}}%
\>[3]{}(\Conid{True},\Conid{False})\mbox{\onelinecomment  are defined and work as expected in Haskell.
}{}\<[E]%
\\
\>[3]{}\mathrel{\wedge}{}\<[18]%
\>[18]{}\mbox{\onelinecomment  \text{\tt \char38{}\char38{}} is for conjunction.
}{}\<[E]%
\\
\>[3]{}\mathrel{\vee}{}\<[18]%
\>[18]{}\mbox{\onelinecomment  \text{\tt \char124{}\char124{}} is for disjunction.
}{}\<[E]%
\\
\>[3]{}\neg {}\<[18]%
\>[18]{}\mbox{\onelinecomment  \text{\tt not} is for negation.
}{}\<[E]%
\ColumnHook
\end{hscode}\resethooks
\begin{hscode}\SaveRestoreHook
\column{B}{@{}>{\hspre}l<{\hspost}@{}}%
\column{3}{@{}>{\hspre}l<{\hspost}@{}}%
\column{E}{@{}>{\hspre}l<{\hspost}@{}}%
\>[3]{}\Conid{True}\mathrel{\wedge}\Conid{False}\equiv \Conid{False}{}\<[E]%
\\
\>[3]{}\Conid{False}\mathrel{\vee}\Conid{True}\equiv \Conid{True}{}\<[E]%
\ColumnHook
\end{hscode}\resethooks
Equality testing is \text{\tt \char61{}\char61{}} (AKA: \ensuremath{\equiv }) and \text{\tt \char47{}\char45{}} (AKA: \ensuremath{\not\equiv }) 
It is a common mistake to forget \text{\tt \char47{}\char61{}} is not equals.
Also equality must be between items of the same type!
Get accustomed to strong typing.  Learn to \emph{love} it.
\\

Let's list some basic functions in Prelude, which is imported by GHCI on startup and you can generally count on being there in your programs (unless you specifically choose not to import it, which you will know how to do if you ever need to make that decision):

\begin{hscode}\SaveRestoreHook
\column{B}{@{}>{\hspre}l<{\hspost}@{}}%
\column{3}{@{}>{\hspre}l<{\hspost}@{}}%
\column{11}{@{}>{\hspre}l<{\hspost}@{}}%
\column{E}{@{}>{\hspre}l<{\hspost}@{}}%
\>[3]{}\Varid{succ}\;\Varid{x}{}\<[11]%
\>[11]{}\mbox{\onelinecomment  is the successor function on x.
}{}\<[E]%
\\
\>[3]{}\Varid{min}\;\Varid{x}\;\Varid{y}\mbox{\onelinecomment  returns the smaller of x and y.
}{}\<[E]%
\\
\>[3]{}\Varid{max}\;\Varid{x}\;\Varid{y}\mbox{\onelinecomment  returns the larger of x and y.
}{}\<[E]%
\ColumnHook
\end{hscode}\resethooks
\newpage 
Since function precedence is highest, these formulae are equivalent:

\begin{hscode}\SaveRestoreHook
\column{B}{@{}>{\hspre}l<{\hspost}@{}}%
\column{3}{@{}>{\hspre}l<{\hspost}@{}}%
\column{4}{@{}>{\hspre}l<{\hspost}@{}}%
\column{12}{@{}>{\hspre}c<{\hspost}@{}}%
\column{12E}{@{}l@{}}%
\column{15}{@{}>{\hspre}l<{\hspost}@{}}%
\column{24}{@{}>{\hspre}l<{\hspost}@{}}%
\column{E}{@{}>{\hspre}l<{\hspost}@{}}%
\>[4]{}\Varid{succ}\;\mathrm{9}{}\<[12]%
\>[12]{}\mathbin{+}{}\<[12E]%
\>[15]{}\Varid{max}\;\mathrm{5}\;\mathrm{4}{}\<[24]%
\>[24]{}\mathbin{+}\mathrm{1}{}\<[E]%
\\
\>[3]{}(\Varid{succ}\;\mathrm{9})\mathbin{+}(\Varid{max}\;\mathrm{5}\;\mathrm{4})\mathbin{+}\mathrm{1}{}\<[E]%
\ColumnHook
\end{hscode}\resethooks
The backtick \text{\tt \char96{}} is in the upper right corner of your keyboard, above the squiqley tilde \text{\tt \char126{}} sign.  We can make any function infix with backticks \text{\tt \char96{}}

\begin{hscode}\SaveRestoreHook
\column{B}{@{}>{\hspre}l<{\hspost}@{}}%
\column{3}{@{}>{\hspre}l<{\hspost}@{}}%
\column{E}{@{}>{\hspre}l<{\hspost}@{}}%
\>[3]{}\mathrm{10}\mathbin{`\Varid{max}`}\mathrm{9}\equiv \Varid{max}\;\mathrm{10}\;\mathrm{9}{}\<[E]%
\\
\>[3]{}\mathrm{92}\mathbin{\Varid{`div`}}\mathrm{10}\equiv \Varid{div}\;\mathrm{92}\;\mathrm{10}{}\<[E]%
\ColumnHook
\end{hscode}\resethooks
\section{Baby's First Functions}

\begin{hscode}\SaveRestoreHook
\column{B}{@{}>{\hspre}l<{\hspost}@{}}%
\column{3}{@{}>{\hspre}l<{\hspost}@{}}%
\column{28}{@{}>{\hspre}l<{\hspost}@{}}%
\column{E}{@{}>{\hspre}l<{\hspost}@{}}%
\>[3]{}\Varid{doubleMe}\;\Varid{x}\mathrel{=}\Varid{x}\mathbin{+}\Varid{x}{}\<[E]%
\\
\>[3]{}\Varid{doubleUs}\;\Varid{x}\;\Varid{y}\mathrel{=}\Varid{x}\mathbin{*}\mathrm{2}\mathbin{+}\Varid{y}\mathbin{*}\mathrm{2}{}\<[E]%
\\
\>[3]{}\Varid{doubleSmallNumber}\;\Varid{x}\mathrel{=}\mathbf{if}\;\Varid{x}\mathbin{>}\mathrm{100}{}\<[E]%
\\
\>[3]{}\hsindent{25}{}\<[28]%
\>[28]{}\mathbf{then}\;\Varid{x}{}\<[E]%
\\
\>[3]{}\hsindent{25}{}\<[28]%
\>[28]{}\mathbf{else}\;\Varid{x}\mathbin{*}\mathrm{2}{}\<[E]%
\\
\>[3]{}\Varid{doubleSmallNumber'}\;\Varid{x}\mathrel{=}(\mathbf{if}\;\Varid{x}\mathbin{>}\mathrm{100}\;\mathbf{then}\;\Varid{x}\;\mathbf{else}\;\Varid{x}\mathbin{*}\mathrm{2})\mathbin{+}\mathrm{1}{}\<[E]%
\ColumnHook
\end{hscode}\resethooks
Remember in Haskell's \text{\tt if} statement the \text{\tt else} is mandatory!
The expression must evaluate to something of the appropriate type. \\
\\
' (apostrophe) is valid in function names, which is pretty cool.

\begin{hscode}\SaveRestoreHook
\column{B}{@{}>{\hspre}l<{\hspost}@{}}%
\column{3}{@{}>{\hspre}l<{\hspost}@{}}%
\column{E}{@{}>{\hspre}l<{\hspost}@{}}%
\>[3]{}\Varid{conanO'Brien}\mathrel{=}\text{\tt \char34 It's~a-me,~Conan~O'Brien!\char34}{}\<[E]%
\ColumnHook
\end{hscode}\resethooks
\section{An Intro to Lists}

Haskell lists are homogenous data structures.  
They store several elements of the same type.

\begin{hscode}\SaveRestoreHook
\column{B}{@{}>{\hspre}l<{\hspost}@{}}%
\column{3}{@{}>{\hspre}l<{\hspost}@{}}%
\column{E}{@{}>{\hspre}l<{\hspost}@{}}%
\>[3]{}\Varid{lostNumbers}\mathrel{=}[\mskip1.5mu \mathrm{4},\mathrm{8},\mathrm{15},\mathrm{16},\mathrm{23},\mathrm{43}\mskip1.5mu]{}\<[E]%
\ColumnHook
\end{hscode}\resethooks
Just so you know, if we had typed the previous in GHCI we would need \text{\tt let}

\begin{hscode}\SaveRestoreHook
\column{B}{@{}>{\hspre}l<{\hspost}@{}}%
\column{3}{@{}>{\hspre}l<{\hspost}@{}}%
\column{E}{@{}>{\hspre}l<{\hspost}@{}}%
\>[3]{}\mathbf{let}\;\Varid{lostNumbers'}\mathrel{=}[\mskip1.5mu \mathrm{4},\mathrm{8},\mathrm{15},\mathrm{16},\mathrm{23},\mathrm{43}\mskip1.5mu]{}\<[E]%
\ColumnHook
\end{hscode}\resethooks
Lists are concatenated with the \text{\tt \char43{}\char43{}} (AKA: \ensuremath{\plus }) operator.

\begin{hscode}\SaveRestoreHook
\column{B}{@{}>{\hspre}l<{\hspost}@{}}%
\column{3}{@{}>{\hspre}l<{\hspost}@{}}%
\column{E}{@{}>{\hspre}l<{\hspost}@{}}%
\>[3]{}[\mskip1.5mu \mathrm{1},\mathrm{2},\mathrm{3},\mathrm{4}\mskip1.5mu]\plus \Varid{lostNumbers}{}\<[E]%
\\
\>[3]{}[\mskip1.5mu \mathrm{1},\mathrm{2},\mathrm{3},\mathrm{4}\mskip1.5mu]\plus [\mskip1.5mu \mathrm{5},\mathrm{6},\mathrm{7},\mathrm{8}\mskip1.5mu]{}\<[E]%
\\
\>[3]{}\text{\tt \char34 hello\char34}\plus \text{\tt \char34 ~\char34}\plus \text{\tt \char34 world\char34}\equiv \text{\tt \char34 hello~world\char34}{}\<[E]%
\ColumnHook
\end{hscode}\resethooks
Yes, strings are just lists (of type \ensuremath{\Conid{Char}}) \\
\\
\ensuremath{\plus } is a fairly expensive operation.  The whole list must be walked through.
\ensuremath{\mathbin{:}} on the other hand (cons) is cheap.  Just tack something on the front.

\begin{hscode}\SaveRestoreHook
\column{B}{@{}>{\hspre}l<{\hspost}@{}}%
\column{3}{@{}>{\hspre}l<{\hspost}@{}}%
\column{E}{@{}>{\hspre}l<{\hspost}@{}}%
\>[3]{}\text{\tt 'A'}\mathbin{:}\text{\tt \char34 ~SMALL~CAT\char34}\equiv \text{\tt \char34 A~SMALL~CAT\char34}{}\<[E]%
\\
\>[3]{}\mathrm{5}\mathbin{:}[\mskip1.5mu \mathrm{1},\mathrm{2},\mathrm{3},\mathrm{4}\mskip1.5mu]\equiv [\mskip1.5mu \mathrm{5},\mathrm{1},\mathrm{2},\mathrm{3},\mathrm{4}\mskip1.5mu]{}\<[E]%
\ColumnHook
\end{hscode}\resethooks
Actually, [1,2,3] is syntactic sugar for 1:2:3:[] \\
\text{\tt \char33{}\char33{}} is the index operator.  

\begin{hscode}\SaveRestoreHook
\column{B}{@{}>{\hspre}l<{\hspost}@{}}%
\column{3}{@{}>{\hspre}l<{\hspost}@{}}%
\column{E}{@{}>{\hspre}l<{\hspost}@{}}%
\>[3]{}\text{\tt \char34 Joe\char34}\mathbin{!!}\mathrm{2}\equiv \text{\tt 'e'}{}\<[E]%
\ColumnHook
\end{hscode}\resethooks
Lists are indexed from 0, like anyone sane would expect.
Lists can be nested (contain other lists). 
They can be different lengths but not different types.
\\ \\
Lists are compared using \ensuremath{\mathbin{<}},\ensuremath{\mathbin{>}},\ensuremath{\leq },\ensuremath{\geq },\ensuremath{\equiv },\ensuremath{\not\equiv } in lexicographic order.

\begin{hscode}\SaveRestoreHook
\column{B}{@{}>{\hspre}l<{\hspost}@{}}%
\column{3}{@{}>{\hspre}l<{\hspost}@{}}%
\column{E}{@{}>{\hspre}l<{\hspost}@{}}%
\>[3]{}[\mskip1.5mu \mathrm{3},\mathrm{2},\mathrm{1}\mskip1.5mu]\mathbin{>}[\mskip1.5mu \mathrm{1},\mathrm{2},\mathrm{3}\mskip1.5mu]\equiv \Conid{True}{}\<[E]%
\ColumnHook
\end{hscode}\resethooks
Useful list operators:
\ensuremath{\Varid{head}}, \ensuremath{\Varid{tail}}, \ensuremath{\Varid{last}}, \ensuremath{\Varid{init}}, \ensuremath{\Varid{length}}, \ensuremath{\Varid{null}}, \ensuremath{\Varid{reverse}},
\ensuremath{\Varid{maximum}}, \ensuremath{\Varid{minimum}}, \ensuremath{\Varid{sum}}, \ensuremath{\Varid{product}}, 
\ensuremath{\Varid{elem}\;\Varid{x}\;\Varid{xs}} (AKA: \text{\tt x~\char39{}elem\char39{}~xs} or \ensuremath{\Varid{x}\in \Varid{xs}})

\begin{hscode}\SaveRestoreHook
\column{B}{@{}>{\hspre}l<{\hspost}@{}}%
\column{3}{@{}>{\hspre}l<{\hspost}@{}}%
\column{E}{@{}>{\hspre}l<{\hspost}@{}}%
\>[3]{}\Varid{init}\;[\mskip1.5mu \mathrm{1},\mathrm{2},\mathrm{3}\mskip1.5mu]\equiv [\mskip1.5mu \mathrm{1},\mathrm{2}\mskip1.5mu]{}\<[E]%
\\
\>[3]{}\Varid{null}\;[\mskip1.5mu \mskip1.5mu]\equiv \Conid{True}{}\<[E]%
\ColumnHook
\end{hscode}\resethooks
\\
\text{\tt take~x~xs} (where x is an Int, takes x elements from xs) \\
\text{\tt drop~x~xs} (where x is an Int, drops the first x elements from xs)

\begin{hscode}\SaveRestoreHook
\column{B}{@{}>{\hspre}l<{\hspost}@{}}%
\column{3}{@{}>{\hspre}l<{\hspost}@{}}%
\column{E}{@{}>{\hspre}l<{\hspost}@{}}%
\>[3]{}\Varid{take}\;\mathrm{3}\;[\mskip1.5mu \mathrm{1},\mathrm{2},\mathrm{3},\mathrm{4},\mathrm{5}\mskip1.5mu]\equiv [\mskip1.5mu \mathrm{1},\mathrm{2},\mathrm{3}\mskip1.5mu]{}\<[E]%
\\
\>[3]{}\Varid{drop}\;\mathrm{3}\;[\mskip1.5mu \mathrm{1},\mathrm{2},\mathrm{3},\mathrm{4},\mathrm{5}\mskip1.5mu]\equiv [\mskip1.5mu \mathrm{4},\mathrm{5}\mskip1.5mu]{}\<[E]%
\ColumnHook
\end{hscode}\resethooks
\text{\tt \char46{}\char46{}} is the range operator and is mega-useful.

\begin{hscode}\SaveRestoreHook
\column{B}{@{}>{\hspre}l<{\hspost}@{}}%
\column{3}{@{}>{\hspre}l<{\hspost}@{}}%
\column{E}{@{}>{\hspre}l<{\hspost}@{}}%
\>[3]{}[\mskip1.5mu \mathrm{1}\mathinner{\ldotp\ldotp}\mathrm{10}\mskip1.5mu]\equiv [\mskip1.5mu \mathrm{1},\mathrm{2},\mathrm{3},\mathrm{4},\mathrm{5},\mathrm{6},\mathrm{7},\mathrm{8},\mathrm{9},\mathrm{10}\mskip1.5mu]{}\<[E]%
\\
\>[3]{}[\mskip1.5mu \text{\tt 'a'}\mathinner{\ldotp\ldotp}\text{\tt 'z'}\mskip1.5mu]\equiv \mbox{\onelinecomment  the list of the lowercase alphabet --
}{}\<[E]%
\ColumnHook
\end{hscode}\resethooks
There is a special form with step values.

\begin{hscode}\SaveRestoreHook
\column{B}{@{}>{\hspre}l<{\hspost}@{}}%
\column{3}{@{}>{\hspre}l<{\hspost}@{}}%
\column{E}{@{}>{\hspre}l<{\hspost}@{}}%
\>[3]{}[\mskip1.5mu \Conid{FIRST},\Conid{SECOND}\mathinner{\ldotp\ldotp}\Conid{LAST}\mskip1.5mu]{}\<[E]%
\ColumnHook
\end{hscode}\resethooks
\begin{hscode}\SaveRestoreHook
\column{B}{@{}>{\hspre}l<{\hspost}@{}}%
\column{3}{@{}>{\hspre}l<{\hspost}@{}}%
\column{28}{@{}>{\hspre}l<{\hspost}@{}}%
\column{54}{@{}>{\hspre}l<{\hspost}@{}}%
\column{E}{@{}>{\hspre}l<{\hspost}@{}}%
\>[3]{}[\mskip1.5mu \mathrm{2},\mathrm{4}\mathinner{\ldotp\ldotp}\mathrm{20}\mskip1.5mu]\equiv [\mskip1.5mu \mathrm{2},\mathrm{4},\mathrm{6},\mathrm{8},\mathrm{10},\mathrm{12},\mathrm{14},\mathrm{16},\mathrm{18},\mathrm{20}\mskip1.5mu]{}\<[E]%
\\
\>[3]{}[\mskip1.5mu \mathrm{3},\mathrm{6}\mathinner{\ldotp\ldotp}\mathrm{20}\mskip1.5mu]\equiv [\mskip1.5mu \mathrm{3},\mathrm{6},\mathrm{9},\mathrm{12},\mathrm{15},\mathrm{18}\mskip1.5mu]{}\<[E]%
\\
\>[3]{}[\mskip1.5mu \mathrm{4},\mathrm{10}\mathinner{\ldotp\ldotp}\mathrm{20}\mskip1.5mu]\equiv [\mskip1.5mu \mathrm{4},\mathrm{10},\mathrm{16}\mskip1.5mu]{}\<[28]%
\>[28]{}\mbox{\onelinecomment  step by 6
}{}\<[E]%
\\
\>[3]{}[\mskip1.5mu \mathrm{20},\mathrm{19}\mathinner{\ldotp\ldotp}\mathrm{10}\mskip1.5mu]\equiv [\mskip1.5mu \mathrm{20},\mathrm{19},\mathrm{18},\mathrm{17},\mathrm{16},\mathrm{15},\mathrm{14},\mathrm{13},\mathrm{12},\mathrm{11},\mathrm{10}\mskip1.5mu]{}\<[54]%
\>[54]{}\mbox{\onelinecomment  step is necessary here!
}{}\<[E]%
\ColumnHook
\end{hscode}\resethooks
Infinite lists are both fine and really cool.

\begin{hscode}\SaveRestoreHook
\column{B}{@{}>{\hspre}l<{\hspost}@{}}%
\column{3}{@{}>{\hspre}l<{\hspost}@{}}%
\column{E}{@{}>{\hspre}l<{\hspost}@{}}%
\>[3]{}[\mskip1.5mu \mathrm{1}\mathinner{\ldotp\ldotp}\mskip1.5mu]\mbox{\onelinecomment  is the natural numbers (if we begin at 1) or positive integers
}{}\<[E]%
\\
\>[3]{}[\mskip1.5mu \mathrm{0}\mathinner{\ldotp\ldotp}\mskip1.5mu]\mbox{\onelinecomment  is the natural numbers if you consider 0 natural
}{}\<[E]%
\\
\>[3]{}[\mskip1.5mu \mathrm{13},\mathrm{26}\mathinner{\ldotp\ldotp}\mskip1.5mu]\mbox{\onelinecomment  is the postive multiples of 13
}{}\<[E]%
\\
\>[3]{}\Varid{take}\;\mathrm{24}\;[\mskip1.5mu \mathrm{13},\mathrm{26}\mathinner{\ldotp\ldotp}\mskip1.5mu]\mbox{\onelinecomment  is the first 24 positive multiples of 13.
}{}\<[E]%
\ColumnHook
\end{hscode}\resethooks
\text{\tt cycle} takes a list and replicates its elements infinitely

\begin{hscode}\SaveRestoreHook
\column{B}{@{}>{\hspre}l<{\hspost}@{}}%
\column{3}{@{}>{\hspre}l<{\hspost}@{}}%
\column{E}{@{}>{\hspre}l<{\hspost}@{}}%
\>[3]{}\Varid{take}\;\mathrm{10}\;(\Varid{cycle}\;[\mskip1.5mu \mathrm{1},\mathrm{2},\mathrm{3}\mskip1.5mu])\equiv [\mskip1.5mu \mathrm{1},\mathrm{2},\mathrm{3},\mathrm{1},\mathrm{2},\mathrm{3},\mathrm{1},\mathrm{2},\mathrm{3},\mathrm{1}\mskip1.5mu]{}\<[E]%
\\
\>[3]{}\Varid{take}\;\mathrm{11}\;(\Varid{cycle}\;\text{\tt \char34 LOL~\char34})\equiv \text{\tt \char34 LOL~LOL~LOL\char34}{}\<[E]%
\ColumnHook
\end{hscode}\resethooks
\section{I'm a List Comprehension}

\end{document}
